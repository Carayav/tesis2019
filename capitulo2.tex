%!TEX root = templateICI.tex
\chapter{Marco Conceptual y Estado del Arte}

En este capítulo en la sección~\ref{sc:MC} deben presentar el Marco Conceptual,
así como en la sección~\ref{sc:EA} es expuesto el estado del arte.

\section{Marco Conceptual}
\label{sc:MC}

\subsection{Ambito Oseo}

Hueso:

Corteza Cortical:

Porosidad
Espesor

\subsection{Ambito Ultrasonido}

\paragraph{Transformada de Fourier:}
La Transformada de Fourier toma una función definida en el dominio del tiempo o
espacio y la transforma al dominio de la frecuencia, que provee
un ambiente natural para el estudio de muchos problemas.
Las técnicas del Análisis de Fourier son usadas en distintos y diversos campos
como, por ejemplo el procesamiento de señales, astronomía, geodésica,
las imágenes medicas y el análisis de voz.\cite{EMS}

\paragraph{Guia de Onda:}
Es una estructura que guía ondas, como las electromagnéticas o mecánicas,
con una pérdida mínima de energía al restringir la expansión a una o dos dimensiones.

\paragraph{Ultrasonido:}
El ultrasonido es una onda mecánica con una frecuencia
para uso clínico entre 1 y 15 MHz, estas frecuencias son mayores al limite
superior audible por un humano (20 KHz). La longitud de onda del ultrasonido en
los tejido es entre ~0.1 y 1.5 mm.
\cite{IMIP}
\cite{AbuZidan2011ClinicalUP}.
Las ondas de ultrasonido son producidas por un transductor

\paragraph{Señal: }
Una señal es una función de uno o más variables independientes, que contienen
información sobre el comportamiento o la naturaleza de algún fenómeno.
\cite{oppenheim2016signals}

\paragraph{Procesamiento Digital de Señales:} El procesamiento digital de señales
es el uso del procesamiento digital, por ejemplo, en computadoras o procesadores
de señales digitales más especializados, para realizar una amplia variedad de
operaciones de procesamiento de la señal. Las señales procesadas de esta manera
son una secuencia de números que representan muestras de una variable continua
en un dominio como el tiempo, el espacio o la frecuencia.

\paragraph{Transmision Axial: }


\paragraph{Descomposición en valores singulares: }

Probe

Emisor

Receptor





Transformadad de Fourier en el tiempo

Transformadad de Fourier espacio-temporal

Transformada Disreta de Fourier

Matriz de transmision

Gel Acoplamiento

Sistema
Emision-ScatMedium-Recepcion

Respuesta al impulse

Transpose
Conjugacion

Subespacio Señal

Subespacio Ruido



Un marco conceptual es una sección de un texto escrito en el ámbito académico que detalla los modelos teóricos, conceptos, argumentos e ideas que se han desarrollado en relación con un tema\footnote{http://comunicacionacademica.uc.cl/images/recursos/espanol/escritura/recurso\_en\_pdf\_extenso/15\_Como\_elaborar\_un\_marco\_conceptual.pdf}.


Si bien no existen limitaciones en esta sección, se le recomienda no sobrepasar las 5 páginas.



\section{Estado del Arte}
\label{sc:EA}

El estado del arte es una recopilación crítica de diversos tipos de texto de un área o disciplina, el cual busca tener una visión sobre un problema específico y cómo éste se ha abordado \cite{londono2014guias}.

Por otra parte, es importante que busque en base de datos especializadas, tales como \textit{IEEE Xplore}, \textit{Science Direct}, \textit{Springer}, \textit{ACM Digital Library}, entre otras. Un aspecto importante es que en esta sección sean discutidos entre 15 y 20 trabajos relevantes en su área de trabajo.
