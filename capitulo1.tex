\chapter{Introducci\'on}


\section{Principales contribuciones}

Los huesos son de vital importancia para la salud y la calidad de vida en general\cite{US}, ya que proporcionan al cuerpo funciones estructurales y metabólicas. Las funciones estructurales de los huesos son dar soporte para realizar las acciones mecánicas y entregar protección mecánica a distintos órganos vitales. Metabólicamente, son encargados de la producción de células sanguíneas y ser la reserva de calcio mas grande del cuerpo humano\cite{Lorincz2009}.
\\[5.5pt]
Los huesos se componen principalmente de dos distintos tipos de tejidos, una capa exterior compacta compuesta de hueso cortical que rodea el tejido esponjoso de hueso trabecular\cite{Cooper2016}.
\\[5.5pt]
Los huesos no saludables, sin embargo, tienen un desempeño deficiente en la ejecución de sus funciones, lo que puede tener consecuencias perjudiciales como las fracturas por fragilidad\cite{US}. En Chile las fracturas de caderas  en adultos mayores sus costes y mortalidad equivalen a la suma de costes y mortalidad por enfermedades cardiovasculares y neoplasias\cite{DINAMARCA-MONTECINOS2015}. 
\\[5.5pt]
La densidad mineral osea es el marcador biológico mas usado para predecir el riesgo a fracturas, sin embargo características oseas relacionadas a la fuerza también incluyen propiedades del hueso cortical y trabecular. Hallazgos recientes sugieren que la evaluación de riesgo de fractura también debe incluir una evaluación precisa del hueso cortical\cite{Bala2015}.
\\[5.5pt]
Actualmente la empresa Azalée tiene un dispositvo de sonda multicanal para transmisión axial. La transmisión axial es una técnica cuantitativa de ultrasonido que permite cuantificar el espesor cortical y la porosidad del hueso cortical. 
\\[5.5pt]
La interfaz actualmente implementada despliega el espectro de ondas guiadas en casi tiempo real (con un framerate de 4Hz), siendo el tiempo de respuesta no lo suficientemente menor para el uso requerido, por lo tanto, se propone una revisión de las etapas del algoritmo usado para el análisis de los datos para encontrar secciones que se puedan optimizar. 
\\[5.5pt]


\section{Esctructura del Documento}
\todo{Adecuar a entrega}

El documento posee la siguiente estructura. A continuación, se presenta la Definición del Problema, luego se entrega la Solución Propuesta. Posteriormente se presentan los Objetivos (Generales y Específicos), sigue la Metodología, para terminar con la Planificación y los Recursos a ocupar durante el Trabajo de Título.