\chapter{Marco Conceptual y Estado del Arte}

En este capítulo en la sección~\ref{sc:MC} deben presentar el Marco Conceptual, así como en la sección~\ref{sc:EA} es expuesto el estado del arte. 

\section{Marco Conceptual}
\label{sc:MC}
Un marco conceptual es una sección de un texto escrito en el ámbito académico que detalla los modelos teóricos, conceptos, argumentos e ideas que se han desarrollado en relación con un tema\footnote{http://comunicacionacademica.uc.cl/images/recursos/espanol/escritura/recurso\_en\_pdf\_extenso/15\_Como\_elaborar\_un\_marco\_conceptual.pdf}. 

Si bien no existen limitaciones en esta sección, se le recomienda no sobrepasar las 5 páginas. 



\section{Estado del Arte}
\label{sc:EA}

El estado del arte es una recopilación crítica de diversos tipos de texto de un área o disciplina, el cual busca tener una visión sobre un problema específico y cómo éste se ha abordado \cite{londono2014guias}. 

Por otra parte, es importante que busque en base de datos especializadas, tales como \textit{IEEE Xplore}, \textit{Science Direct}, \textit{Springer}, \textit{ACM Digital Library}, entre otras. Un aspecto importante es que en esta sección sean discutidos entre 15 y 20 trabajos relevantes en su área de trabajo. 